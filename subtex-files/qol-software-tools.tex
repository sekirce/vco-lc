%----------------------------------------------------------------------------------------
% Quality of Life
%----------------------------------------------------------------------------------------

\section{Quality of Life - software tools}

Quality of life regarding LaTex for this document, commercial/proprietary EDA software for this and previous projects and potential open source EDA software for future projects.

%----------------------------------------------------------------------------------------
% LaTeX QoL
%----------------------------------------------------------------------------------------

\subsection{QoL \LaTeX }


\subsubsection{Colored rows and columns}

\href{https://tex.stackexchange.com/questions/94799/how-do-i-color-table-columns}{Coloring inside of table}

% \documentclass{article}
% \usepackage{xcolor,colortbl}

% \newcommand{\mc}[2]{\multicolumn{#1}{c}{#2}}
% \definecolor{Gray}{gray}{0.85}
% \definecolor{LightCyan}{rgb}{0.88,1,1}

% \newcolumntype{a}{>{\columncolor{Gray}}c}
% \newcolumntype{b}{>{\columncolor{white}}c}

% \begin{document} 


\begin{table}[ht!]
    \begin{tabular}{l | a | b | a | b}
        \hline
        \rowcolor{LightCyan}
        \mc{1}{}  & \mc{1}{x} & \mc{1}{y} & \mc{1}{w} & \mc{1}{z} \\
        \hline
        variable 1 & a & b & c & d \\
        variable 2 & a & b & c & d \\ \hline
    \end{tabular}
\end{table}

% \end{document}


Subfigures examples in comments

% \begin{figure}
% \centering
% \begin{subfigure}{.5\textwidth}
% 	\centering
% 	\includegraphics[width=.4\linewidth]{image1}
% 	\caption{A subfigure}
% 	\label{fig:sub1}
% \end{subfigure}%
% \begin{subfigure}{.5\textwidth}
% 	\centering
% 	\includegraphics[width=.4\linewidth]{image1}
% 	\caption{A subfigure}
% 	\label{fig:sub2}
% \end{subfigure}
% \caption{A figure with two subfigures}
% \label{fig:test}
% \end{figure}

% \begin{figure}
% \centering
% \begin{minipage}{.5\textwidth}
% 	\centering
% 	\includegraphics[width=.4\linewidth]{image1}
% 	\captionof{figure}{A figure}
% 	\label{fig:test1}
% \end{minipage}%
% \begin{minipage}{.5\textwidth}
% 	\centering
% 	\includegraphics[width=.4\linewidth]{image1}
% 	\captionof{figure}{Another figure}
% 	\label{fig:test2}
% \end{minipage}
% \end{figure}

\subsubsection{Nested tables using tabular}

\href{https://tex.stackexchange.com/questions/40561/table-with-multiple-lines-in-some-cells}{source:} You could nest a tabular within another tabular:


\begin{tabular}{cccc}
\rowcolor{LightCyan}
  One & Two & Three & Four \\
  Een & Twee & Drie & Vier \\
  One & Two & 
    \begin{tabular}{@{}c@{}}Three \\ Drie\end{tabular}
  & Four
\end{tabular}


\subsection{Open Source EDA Software}

%----------------------------------------------------------------------------------------
% NGSPICE + SG13G2
%----------------------------------------------------------------------------------------

\subsubsection{NGSPICE}

Regarding open-source PDK for SG13G2, it's work in progress, but I think worth to consider. Even if just the simulator can be replaced with ngspice or xyce, it could boost productivity from not being license-bound. Please check this out:

\href{https://github.com/IHP-GmbH/IHP-Open-PDK/tree/main/ihp-sg13g2}{\textbf{ihp sg13g2}}

Documentation on github is sparse, some more info is here:
\href{https://www.ihp-microelectronics.com/events-1/detail/openpdk-opentooling-and-open-source-design-an-initiative-to-push-development}{\textbf{openPDK opeentooling and open source design}}

\href{https://www.ihp-microelectronics.com/services/research-and-prototyping-service/fast-design-enablement/open-source-pdk}{\textbf{open source PDK, research and prototyping}}

\subsubsection{Required libraries - Linux Installation}

libxaw7-dev and libreadline6-dev
\\
KiCad plus ngspice

% \begin{verbatim}
% 	/home/aleksandarvukovi/ngspice-42/examples/osdi/hicuml0/Modelcards/model-card-hicumL0V1p11_mod.lib	
% \end{verbatim}

\subsubsection{openEMS}

openEMS is a free and open source electromagnetic field solver based on the Finite-Difference Time Domain (FDTD) method. Using an improved version of the highly-successful FDTD method (known as Equivalent-Circuit FDTD, or EC-FDTD), openEMS solves Maxwell's equations in discretized space and time to directly simulates the propagation of electromagnetic waves, in a 3D full-wave manner. It has the potential to analyze problems in important applications such as RF/microwave circuit design, antenna, radar, meta-material, and medical research.

\subsection{Commercial EDA software}

\subsubsection*{EMX and Momentum}
EMX software from Cadence, Momentum from ADS

\newpage

\subsubsection{Change font in Visualisation Cadence}

Example:

\begin{verbatim}
	myFont = "Roboto Mono Medium,14,-1,5,75,0,0,0,0,0"
	envSetVal("viva.horizMarker" "font" 'string myFont)
	envSetVal("viva.referenceLineMarker" "font" 'string myFont)
	envSetVal("viva.vertMarker" "font" 'string myFont)
	envSetVal("viva.pointMarker" "font" 'string myFont)
	envSetVal("viva.refPointMarker" "font" 'string myFont)
	envSetVal("viva.specMarker" "font" 'string myFont)
	envSetVal("viva.interceptMarker" "font" 'string myFont)
	envSetVal("viva.circleMarker" "font" 'string myFont)
	envSetVal("viva.interceptMarker" "font" 'string myFont)
	envSetVal("viva.multiDeltaMarker" "font" 'string myFont)
	envSetVal("viva.transEdgeMarker" "font" 'string myFont)

	envSetVal("viva.trace" "lineThickness" 'string "thick")
	envSetVal("viva.trace" "lineStyle" 'string "solid")
	envSetVal("viva.trace" "symbolsOn" 'string "true")
	envSetVal("viva.trace" "symbolStyle" 'string "Diamond")
\end{verbatim}

%----------------------------------------------------------------------------------------
% VerilogA 14bit Decoder
%----------------------------------------------------------------------------------------

\subsubsection{VerilogA Decoder Code}

\begin{file}[Example of 15 bit decoder]
	\begin{lstlisting}
	`include "disciplines.vams"
	module Decoder_15bit_va(L);
	output     [14:0] L;
	electrical [14:0] L;

	parameter real 	tdel 	= 20.0p,
			trise	= 40p,
			tfall	= 40p,
			VDD 	= 1.1;
	parameter integer select = 0;
	integer   out[14:0], mask, i, j;

	analog begin
	
	@(initial_step) begin
		for(i=14; i>=0; i=i-1) begin
		mask=1;
		for(j=1; j<=i; j=j+1) mask=mask*2;
		out[i]=(select & mask) >> i;
		end
	end
	\end{lstlisting}
\end{file}

\begin{file}[Example of 15 bit decoder - cont.]
	\begin{lstlisting}
	V(L[0]) <+ transition(VDD * out[0],
	 tdel, trise, tfall);
	V(L[1]) <+ transition(VDD * out[1],
	 tdel, trise, tfall);
	V(L[2]) <+ transition(VDD * out[2],
	 tdel, trise, tfall);
	V(L[3]) <+ transition(VDD * out[3],
	 tdel, trise, tfall);
	V(L[4]) <+ transition(VDD * out[4],
	 tdel, trise, tfall);
	V(L[5]) <+ transition(VDD * out[5],
	 tdel, trise, tfall);
	V(L[6]) <+ transition(VDD * out[6],
	 tdel, trise, tfall);
	V(L[7]) <+ transition(VDD * out[7],
	 tdel, trise, tfall);
	V(L[8]) <+ transition(VDD * out[8],
	 tdel, trise, tfall);
	V(L[9]) <+ transition(VDD * out[9],
	 tdel, trise, tfall);
	V(L[10]) <+ transition(VDD * out[10],
	 tdel, trise, tfall);
	V(L[11]) <+ transition(VDD * out[11],
	 tdel, trise, tfall);
	V(L[12]) <+ transition(VDD * out[12],
	 tdel, trise, tfall);
	V(L[13]) <+ transition(VDD * out[13],
	 tdel, trise, tfall);
	V(L[14]) <+ transition(VDD * out[14],
	 tdel, trise, tfall);

	end  
	endmodule
	\end{lstlisting}
\end{file}

\newpage

\subsubsection*{Saving Histories in ADE explorer}

\href{https://community.cadence.com/cadence_blogs_8/b/cic/posts/virtuosity-exploring-histories}{Virtuosity: Exploring Histories}
\begin{verbatim}
envSetVal("maestro.explorer" "onHistoryNameCollision" 'cyclic "Overwrite")
envSetVal("adexl.historyNamePrefix" "showNameHistoryForm" 'boolean t)
envSetVal("maestro.explorer" "onHistoryNameCollision" 'cyclic "IncrementAsNew")
\end{verbatim}

%----------------------------------------------------------------------------------------
% PVS Design Flow 
%----------------------------------------------------------------------------------------

\subsubsection*{PVS Design Flow}

PVS has problems with digital circuits using global nets.

\begin{itemize}
	\item PVS DRC 
	\item PVS LVS - you can setup Quantus from here, generates svdb file, do this after it passes LVS
	\item Quantus PVS 
\end{itemize}


\subsubsection*{Curve fitting using Wolfram Alpha}

\href{https://www.reddit.com/r/arduino/comments/12b7t9p/hello_everyone_im_trying_to_get_the_polynomial/}{usefull reddit post}
\href{https://www.wolframalpha.com/input?i=curve+fitting}{Wolfram Alpha without programming}

Curve fitting for Vtune input range (-1.6 ; 1.6) to calculate polynomial:

\begin{verbatim}
{{-1.6, -2.7}, {-1.1, -2.4}, {-0.5, -1.8}, {0, 0 }, {0.5, 1.8}, {1.1, 2.4}, {1.6, 2.7 }}	
\end{verbatim}
% {{-1.6, -2.7}, {-1.1, -2.4}, {-0.5, -1.8}, {0, 0 }, {0.5, 1.8}, {1.1, 2.4}, {1.6, 2.7 }}
Curve fitting for Vtune input range (0 ; 3.2) to calculate polynomial:
\begin{verbatim}
{{0, -2.7}, {0.5, -2.4}, {1.1, -1.8}, {1.6, 0 }, {2.1, 1.8}, {2.7, 2.4}, {3.2, 2.7 }}		
\end{verbatim}

\subsubsection*{VerilogA parse code}

\begin{verbatim}

vmsUpdateCellViews(?lib "elta_VCO_aleksandarv" ?cell "vco_va_model" ?view "veriloga")
\end{verbatim}