
%----------------------------------------------------------------------------------------
%----------------------------------------------------------------------------------------
%----------------------------------------------------------------------------------------
%----------------------------------------------------------------------------------------
\newpage

\section{Old Project}

Old project LO block VCO and VCO Buffer for topcon late 2022. % or time frame

\subsection{Introduction} % Unnumbered section

LO block is LC VCO with internal inductor and fully differential output buffer. Its schematic shall ensure differential connection to the mixer LO input and PLL RF input.

%----------------------------------------------------------------------------------------
\subsection{Technical Requirements}


\begin{table}[ht]
	\centering
	\begin{tabular}{|c|l|c|c|c|c|c|}
		\hline
		\rowcolor{DarkCyan}
		\textbf{\textcolor{white}{\#}} & \textbf{\textcolor{white}{LO Requirements}} &\textbf{\textcolor{white}{Note}}  & \textbf{\textcolor{white}{min}} & \textbf{\textcolor{white}{typ}} & \textbf{\textcolor{white}{max}} & \textbf{\textcolor{white}{Units}} \\
		\hline
		\rowcolor{LightCyan}
		& \multicolumn{6}{c|}{VCO Requirements} \\
		\hline
		1 & Full LO range &  & 6300  &  & 13700 & MHz \\ 
		\hline
		2 & Phase Noise at 300 kHz &  &  &  & <-101 & dBc/Hz \\ 
		\hline
		3 & Vtune &  & 0.1 &  & 1 & V  \\ 
		\hline
		4 & Tuning Sensitivity $K_{VCO}$ &  &  &  & 100 & MHz/V  \\ 
		\hline
		5 & Pushing & TBD &  &  & 2 & MHz/V  \\ 
		\hline
		6 & Output Voltage & TBD & 0.8 &  & & $V_{p-p}$  \\ 
		\hline
		7 & Load Impedance &  &  &  & 100 & fF  \\ 
		\hline
		\rowcolor{LightCyan}
		& \multicolumn{6}{c|}{VCO and output Buffer} \\
		\hline
		8 & Output Voltage & TBD & 0.8  &  &  & V  \\ 
		\hline
		9 & Load Impedance & TBD &  &  & 1000 & fF  \\ 
		\hline
		10 & Harmonic suppression ($2_{nd}$, typ) &  & -15 &  &  & dBc  \\ 
		\hline
		11 & Pulling (14 dB Return Loss, Any Phase) & TBD &  &  & 2 & MHz  \\ 
		\hline
		\rowcolor{LightCyan}
		& \multicolumn{6}{c|}{General Specifications} \\
		\hline
		12 & Operating Temperature Range &  & -40 &  & 125 & °C  \\ 
		\hline
		13 & Supply Voltage &  & 1 & 1.1 & 1.2 & V  \\ 
		\hline
		14 & Supply Current &  &  &  & 20 & mA  \\ 
		\hline
		15 & Shutdown Current &  &  &  & 10 & $\mu$A  \\ 
		\hline
		16 & Time to Switch Between Cores  &  &  & 3 & 5 & ms  \\ 
		\hline
	\end{tabular}
	\label{tab:table-spec}
	\caption{Specification Requirements} 

\end{table}

LO block should have the characteristics given in Table \ref{tab:table-spec}. Due to wide frequency range it is necessary to use switchable capacitor banks. Min and max for supply voltage are not defined in the original document, and time for switching between two cores was defined during the meeting. Currently Full LO range is split between two cores so the requirement should look like this:

\begin{table}[ht]
	\centering
	\begin{tabular}{|c|l|c|c|c|c|c|}
		\hline
		\rowcolor{LightCyan}
		& LO range requirements & Note & min & typ & max & Units \\
		\hline
		1 & High Band core LO range &  & 10000  &  & 13700 & MHz \\ 
		\hline
		2 & Low Band core LO range &  & 6300 &  & 10000 & MHz \\ 
		\hline
	\end{tabular}
	\caption{LO Range Two Core Specification Requirements} 
\end{table}

\begin{question}[\itshape What's the variation for supply voltage?]
	Not defined probably will be the same as the rest of the 
\end{question}


\begin{question}[\itshape $K_{VCO}$ Why is it in the max column? ]
	Didn't get answer for this. It's probably also typical and a value that is expected by the PLL design.
\end{question}

% Footnotes do not work.

Process corners can be found at:

\begin{verbatim}
/tech/tsmc/tsmc40/models/spectre/crn40lp_2d5_v2d0_2_shrink0d9_embedded_usage.scs
\end{verbatim}
% Seems that 

\subsection{Scaldio Design Review}

\begin{question}[\itshape How should calibration be implemented to achieve output voltage peak to peak and minimize noise?]
Is it done for whole PLL?
\end{question}

\begin{question}[\itshape Difference between class-B and class-C vco? ]
	Scaldio uses class-C, so all parasitic capacitances connected to the tail don't matter. Having trouble with observing the currents of drain so cannot check this and compare them. Scaldio looks something between class C and class B because of $V_{gbias}$ voltage and because adding inductor between tail NMOS and switching pair improves phase noise. This is attributed to class B, while class C only needs tail capacitance.
\end{question}

\begin{figure}[ht!]
	\includegraphics[width=\linewidth]{Figures/class_C_vs_class_B.png}
	% \caption{Difference between class C and class B}
	\label{fig:classC_classB}
\end{figure}

Lowest bit of digital varactor (L0\_PLL\_hbVCO\_Cdig\_SC2B\_VCOS\_SC2C\_PLL) doesn't do anything, isn't even monotonous. Maybe makes more sense when simulating extracted cells. Digital varactor needs to be redesigned, it could lower the Q factor.
Main difference between class B and class C can be observed by investigating their drain currents.

\begin{figure}[ht!]
	\includegraphics[width=\linewidth]{Figures/drainCurrent_classB_vs_classC.png}
	% \caption{Difference between class C and class B}
	\label{fig:drainCurrent_classB_vs_classC}
\end{figure}

\subsection{Main testbench}

Main testbench covers everything except for frequency pulling and full LO range. The simulation results are obtained at the higher end of the LO range. Phase noise is simulated by pss+pnoise.

\begin{info} % Information block
	With Noise Type=timeaverage and ALL(AM,PM,USB,LSB), you can plot the AM and PM components as well as the total noise. In addition, you can plot phase noise and FM jitter results for oscillators. Plotting is done using the Direct Plot Form.
	\href{https://community.cadence.com/cadence_blogs_8/b/rf/posts/virtuoso-video-diary-noise-simulation-in-spectre-rf-using-improved-pnoise-hbnoise-and-direct-plot-form-options}{\textbf{External link}}
\end{info}

Function of phase noise is simulated for PM noise type.

How to choose beat frequency for autonomous system from forum \href{https://community.cadence.com/cadence_technology_forums/f/custom-ic-design/2661/beat-frequency-in-spectrerf-pss-simulation}{\textbf{thread}}.

\begin{info} % Information block
In an autonomous system (e.g. an oscillator), you turn on the "oscillator" checkbox, and the beat frequency is then the estimated frequency, which gives PSS a starting point to solve for the oscillator frequency. It's important when in oscillator mode to select the outputs of the circuit, which include any subharmonics. In other words, if you have an oscillator followed by a divider, point at the divider output, and give the estimated divided frequency as the beat frequency. Again, this is because you need to solve an integer number of cycles of all the frequencies in the circuit. Note, don't use oscillator mode for circuits which aren't oscillators, since you're then trying to get the simulator to solve for an unknown which is not unknown, which may lead to convergence problems.
\end{info}

Most of the $I_{bias}$ tune digital control is not used for the higher band, so by increasing the number of steps to cover even higher frequencies than the original design finer bias control is needed.


\subsection{Simulation Results for Scaldio design}

Simulated only nominal corner with change for $V_{DD}$ only for frequency pushing simulation. 

\begin{table}[ht]
	\centering
	\begin{tabular}{|c|l|c|c|c|c|c|c|}
		\hline
		\rowcolor{LightCyan}
		& LO Requirements & Note & min & typ & max & Sim(Typ) & Units \\
		% \hline
		% & \multicolumn{7}{|c|}{VCO Requirements} \\
		\hline
		1 & Phase Noise at 300 kHz &  &  &  & <-101 & -98 & dBc/Hz  \\ 
		\hline
		2 & Tuning Sensitivity $K_{VCO}$ &  &  &  & 100 & over & MHz/V  \\ 
		\hline
		3 & Pushing & TBD &  &  & 2  & 279.7 & MHz/V  \\ 
		\hline
		4 & Output Voltage & TBD & 800 &  & & 809.3 & $mV_{p-p}$  \\ 
		\hline
		5 & Harmonic suppression ($2_{nd}$, typ) &  & -15 &  & & -26.78 & dBc  \\ 
		\hline
		6 & Pulling (14 dB Return Loss, Any Phase) & TBD &  &  & 2  & 19.51 & MHz  \\ 
		\hline
	\end{tabular}
	\label{table-ScaldioResults}
	\caption{Scaldio IMEC design Results}
\end{table}

Phase noise changes a lot for different tuning voltages between -90 and -100 dBc/Hz. Phase noise is probably not modeled ok because the VCO doesn't have the connection between current bias tail and the switching pair of VCO as transmission line. That inductance and $C_{tail}$ should resonate at 2$\omega_O$ (tank oscillating frequency), but only in the case of class B oscillator. Check what type is vco oscillator. %This is hard to check because pss\_tran drain currents look wierd.

\newpage

%----------------------------------------------------------------------------------------
\subsection{Q factor of LC tank}

Need testbenches capacitance of tank, varactors and inductor, and for different controls and also corners. Process corners for varactors are the same as MOSFET.

% and of tail (to see how to make it resonate at the double of the oscillating frequency).

\begin{question}[\itshape What kind of chip is it?]
	What kind of inductance is expected to be connected on $V_{DD}$ and $V_{SS}$ pins. This may or may not change the results. Removed all together right now.
\end{question}


\begin{figure}[ht!]
	\includegraphics[width=\linewidth]{Figures/QL_inductor.png}
	\caption{Q factor and inductance of L}
	\label{fig:qlinductor}
\end{figure}
Testbench for differential Q and L of the inductor that is EMX simulated is shown in Figure \ref{fig:qlinductor}.

Q factor is better at the higher frequency.

Simulated Q factor of both varactors and inductor. Q factor of digital varactor for lower bit controls is not much better than Q of inductor.

% Differential Q factor of an inductor 

% Zdiff = Z11 + Z22 - Z12 - Z21
% Ldiff = Imag {Zdiff} / 2πf
% Qdiff = Imag{Zdiff } / Real{Zdiff }

Capacitor calculating capacitance and Q factor:

\begin{equation}
	Y_{diff} (im) = imag(ypm('sp 1 1)) + imag(ypm('sp 2 2)) - imag(ypm('sp 2 1)) - imag(ypm('sp 1 2))
\end{equation}

\begin{equation}
	Y_{diff} (re) = real(ypm('sp 1 1)) + real(ypm('sp 2 2)) - real(ypm('sp 2 1)) - real(ypm('sp 1 2))
\end{equation}

Capacitance:
\begin{equation}
	C_{diff} = \dfrac{Y_{diff} (im)}{2\pi f}
\end{equation}

Q factor of capacitor:
\begin{equation}
	Q_C = \dfrac{Y_{diff} (im)}{ Y_{diff} (re)}
\end{equation}

\begin{figure}[h!]
	\includegraphics[width=\linewidth]{Figures/Dvaractor.png}
	\caption{C and Q of digital varactor through controls}
	\label{fig:dvaracator}
\end{figure}

More info can be found in  \href{https://ieeexplore.ieee.org/abstract/document/5537949}{\textbf{paper }} A Thorough Analysis of the Tank Quality Factor in LC Oscillators with Switched Capacitor Banks.

Found definitions:

\begin{equation}
	Q_{LC} = \dfrac{1}{R}\sqrt{\dfrac{L}{C}} = \frac{f_r}{\Delta f} = \frac{\omega _r}{\Delta \omega} = \dfrac{\tau _d \omega}{2}
\end{equation}

where $\tau _d$ is group delay.

\begin{question}[\itshape How to calculate group delay using sparam analyis?]
	?
\end{question}

\subsection{Calculating Q factor‚ LC tank by ringing method}

Link to \href{https://www.giangrandi.ch/electronics/ringdownq/ringdownq.shtml}{\textbf{website}} that shows ring down method of calculating Q factor. Not implemented in virtuoso TODO.


%----------------------------------------------------------------------------------------
\subsubsection{Figure of Merit different definitions}

\begin{info} % Information block
	The theoretical maxima for the FoM of an oscillator is given by $FoM=174+20log_{10}(Q)$ dBc/Hz
\end{info}

How to calculate oscillator FoM? Is $Q_L$ factor dominant enough to just equate it with LC tank Q facor.

The widely used FOM is calculated by:
\begin{equation}
	FoM = L\{ \Delta \omega \}( \Delta f/f_0)^2 PVCO [mW]
\end{equation}

Where $L\{\Delta \omega\}$ is the phase noise, $ \Delta \frac{f}{f_0}$ is the ratio between the offset frequency and the carrier, and $P_{VCO}$ is the power consumption of the VCO-core. There is also $FoM^T$ and $FoM_A$ 

\begin{equation}
	FoM^T = FoM - 20 \log (\dfrac{FTR}{10})
\end{equation}

and 

\begin{equation}
	FoM_A = FoM + 10 \log (A)
\end{equation}

where A is area [$mm^2$], and FTR frequency tuning range [$\%$]

Information about digital varactor copied from \href{https://www.doe.carleton.ca/~ddchen/Tutorials/DCO.pdf}{\textbf{External link}}

\begin{info} % Information block
Each bit of MIM varactor contains two MIM capacitors connected differentially with a series switch, two pull-up and two pull-down transistors to effectively turn the varactor between its high and low-capacitance states. Measured intrinsic Q of the MIM capacitor is 80 at 3.6 GHz. When is turned on, i.e., high-capacitance state, the varactor Q drops to 30. When is turned off to be in low-capacitance state, the parasitic capacitance of the MIM capacitor and transistors has an effective of 50. The pull-down transistors set the DC levels for drain and source of at 0 V so that
can be efficiently turned into triode region while the weak pull-up transistors set the DC level to VDDOSC to reduce the parasitic capacitance of thus increasing of the parasitic capacitance. The pull-up pMOS can be implemented by either resistors or transistors. The latter was chosen for silicon area efficiency. Compared to MOS varactors, MIM varactors have a much lower . However, since the differential phase-stability inductor is only 10, the impact of lower varactor is tol erable. When the MIM varactor is at its low-capacitance state, the large DCO internal signal swing and the DC level of 1.4-V supply voltage at source/drain of the pull-up transistors force the drain -nwell junction diodes of the pull-up pMOS to momentarily go into forward-bias condition resulting in a latch-up concern. However, since the forward-bias condition occurs only in 50\% of a 3–4 GHz period, the latch-up phenomenon with the parasitic BJTs can not be triggered.
\end{info}

%----------------------------------------------------------------------------------------
\subsection{Lowering frequency pushing and LDO}

About Frequency pushing from \href{https://www.atlantis-press.com/article/6376.pdf}{\textbf{this paper}}

\begin{info} % Information block
An LC-tank VCO circuit has been implemented in a standard 0.35 $\mu m$ CMOS technology. It is based on a two-transistor biasing structure that improves the performance of frequency pushing and frequency tuning range. Final measurement of proposed structure gives 516 MHz tuning range with 2.278 GHz center frequency and about 0.55$\%$/V frequency pushing in the worst case. The achieved FOM is about -180dBc/Hz, which is very close to the simulated value. This structure is proven to be particularly suitable for achieving low FOM in the VCO circuits having low Q factor LC-tank. Both, the proposed structure and the FOM optimization method, can also be applied to the VCO designs for the applications at higher frequencies, such as 5GHz VCOs for Wireless LAN applications.
\end{info}

\begin{question}[\itshape Can VCO work for lower voltage of 0.9 V?]
	This may be needed if LDO is required because of frequency pushing?
\end{question}

\begin{question}[\itshape Does frequency only happen because of the ripple directly induced by buffers e.g.?]
	Or could it happen because of EM crosstalk?
\end{question}


Is frequency pushing testbench good? Look into this paper. % which paper
Different testbench would be to make a transient change in Vdd and check how much frequency changes.


Results for class C with pmos bias, only dc change of $V_{DD}$:

\begin{center}
	\begin{tabular}{|l|c|c|c|c|c|c|c|c|c|}
		\hline
		Parameter & typical & spec  & min & max & ss -40 & ss 125 & ff -40 & ff 125 & Units \\
		% \hline
		% & \multicolumn{7}{|c|}{VCO Requirements} \\
		\hline
		Frequency Pushing & 43.93 & < 2 &  43.93 & 100.1 & 100.1 & 64.36 & 75.48 & 64.09 & MHz \\ 
		\hline
		Frequency  & 14.6 & > 14.2  & 14.6 & 15.83 & 15.54 & 15.48 & 15.83 & 15.59 & GHz  \\ 
		\hline
	\end{tabular}
\end{center}

This shows some improvement from frequency pushing of 250 - 300 MHz that was observed for the nmos bias of IMEC Scaldio design.

%----------------------------------------------------------------------------------------
\subsection{Changing topology and lowering phase noise}

Decision on why is single sided oscillator chosen instead of double sided oscillator.

\begin{info}
However, if non-negligible parasitic capacitances are found at the tank outputs, the phase-noise performance of the DS-VCO may be seriously degraded, while that of the SS-VCO remains unaffected.
\end{info}

More on the $\frac{1}{f^2}$ Phase Noise Performance of CMOS Differential-Pair LC-Tank Oscillators in \href{https://backend.orbit.dtu.dk/ws/files/3913656/Andreani.pdf}{\textbf{paper}} by Pietro Andreani.


\subsection{Hybrid class C and class B}

\begin{info}
Further, the proposed VCO solves the issue of the hybrid mixed-signal start-up procedure exposed in [8]. The main drawback of this approach is that, if oscillation stops for some unaccountable reason, the VCO can only be restarted actuating again the whole start-up procedure.
\end{info}

Referenced paper is:

\begin{itemize}
	\item [8] J. Chen, F. Jonsson, M. Carlsson, C. Hedenas, and L.-R. Zheng, “\href{https://ieeexplore.ieee.org/document/5951800}{A low power, startup ensured and constant amplitude class-C VCO in \SI{0.18}{\micro\metre} CMOS},” IEEE Microw. Wireless Compon. Lett., vol. 21, no. 8, pp. 427–429, 2011. Dec. 2008.
\end{itemize}


\subsection{Enhanced Oscillation Swing}

Class-C VCO With Amplitude Feedback Loop for Robust Start-Up and Enhanced Oscillation Swing in \href{https://ieeexplore.ieee.org/stamp/stamp.jsp?arnumber=6377236}{\textbf{this paper.}} Phase noise is lowered by lowering $V_{gbias}$, but it makes oscillations start up harder.

\begin{info}
As noted above, the phase noise improves with increasing the oscillation amplitude, which here would mean lowering the gate bias voltage, $V_{bias}$ . Unfortunately, the original class-C oscillator limits the fixed $V_{bias}$ from being set low enough, otherwise the oscillation may not start up. In [11], a high-swing class-C (HSCC) oscillator was introduced, which removed the tail current transistor of the original class-C oscillator [6]. Instead, an automatic amplitude control was introduced to stabilize the oscillation amplitude. In this work, instead of the transformer used in [11], we choose a simple RC bias circuit.
\end{info}

This is from \href{https://www.semanticscholar.org/paper/Dual-Core-High-Swing-Class-C-VCO-design-Kim-Kim/c9551af0809604f76263af49976df9efc213bb8e}{\textbf{paper}} Dual-Core High-Swing Class-C VCO design, and references 


\begin{itemize}
	\item [6] A. Mazzanti and P. Andreani, “\href{https://ieeexplore.ieee.org/document/4684621}{Class-C harmonic CMOS
	VCOs, with a general result on phase noise},” IEEE J.
	Solid-State Circuits, vol. 43, no. 12, pp. 2716–2729,
	Dec. 2008.
	\item [11] M. Tohidian, A. Fotowat-Ahmadi, M. Kamarei, and F.
	Ndagijimana, “\href{https://ieeexplore.ieee.org/document/6045015}{High-swing class-C VCO},” in Proc.
	ESSCIRC, Sep. 2011, pp. 495–498
\end{itemize}


\subsection{Questions about new double feedback}


Amplitude Feedback for Robust start up

% \begin{question}[\itshape What is?]
% 	This may be needed if LDO is required because of frequency pushing?
% \end{question}


\subsection{OTA1 - Start up  and gate voltage bias control}

Referent voltage should be set and controlled around 800 mV. Need tests for OTAs inside for VCO, currently simulated only PM and DC gain, should test different $V_{ref}$ levels.

\subsection{OTA2 - Start up and bias control}

Referent voltage should be set and controlled around 400 mV. Problem with OTA2 loop is amplifying noise into the tail bias current. So the new problem arises as noise shaping in LOOP2 is needed. If a OTA of low uGBW is used than start up is too slow.


%----------------------------------------------------------------------------------------
\subsection{Full LO Range and Frequency Recentering}

% TODO Setup the lb vco, and also setup different testbenches for the highest and the lowest frequency available. Add corner analysis PVT.

TODO Add corners for \textbf{fs sf} and similar. Because the LO range is split on two cores, ideally halved. Results are simulated for process and temperature corners: 

\begin{table}[ht]
	\centering
	\begin{tabular}{|l|c|c|c|c|c|}
		\hline
		Core and Specifictaion & min & max & sim(min) & sim(max) & Units \\
		\hline
		\multicolumn{6}{|c|}{VCO core split} \\
		\hline
		High Band Higher limit & 13700 &  & 15580 & 16190  &  MHz  \\ 
		\hline
		High Band Lower limit &  & 10000 & 10870 & 12460 &  MHz  \\ 
		\hline
		High Band range & 3700 &  & 4710 & 3730 &  MHz  \\ 
		\hline
		Low Band Higher limit & 10000 &   & 13230 & 14120 &  MHz  \\ 
		\hline
		Low Band Lower limit &  & 6300 & 6724 & 8028  &  MHz  \\ 
		\hline
		Low Band range & 3700 &  & 6506 & 6092 &  MHz  \\ 
		\hline
	\end{tabular}
	\caption{LO range specification and simulation for Scaldio LC tank}
\end{table}

Expecting the drop for schematic simulated frequency range, covered range should at least be larger than needed when recentered: 

\begin{equation}
	HBFR = 13700 - 10000 = 3700 < 3730 = 16190 - 12460
\end{equation}

For lower band it's similarily calculated

\begin{equation}
	HBFR = 10000 - 6300 = 3700 < 6092 = 14120 - 8028
\end{equation}

Lower and higher band are assymetrical and they overlap for at least:

\begin{equation}
	HBLBoverlap_{high} = 13230 - 10870 = 2360
\end{equation}

\begin{equation}
	HBLBoverlap_{low} = 14120 - 12460 = 1660
\end{equation}

The worst available frequency digital controlled range is $6092 + 3730 - 1660 = 8162$ MHz which is higher than 7400 MHz.

\begin{question}[\itshape Are the two VCO-s in the same process corner at the same time?]
	Yes. If not than the calculations are wrong.
\end{question}

NOTE: Analog varactor 0-Vt-1 change does not work so it wasn't included. Further frequency recentering after extraction will be needed.

%----------------------------------------------------------------------------------------
\subsection{Pulling testbench and design of VCO buffer}

Needs port at and tuning circuit to keep the reflection at -14 dB. Port of reference impedance 10 k$\Omega$ and portAdapter from rfExamples. S parameter analysis to show if the reflection really is -14 dB? Load impedance increased from 100 fF to 1000 fF. VCO buffer is also needed because of the frequency pulling. Does each core have a buffer or do they share it? first make a buffer than maybe a question.

By adding two buffers from \verb c LO_FDDQ_v6c, \verb c LO_FDDQ_INHB_v5_JCx_scaldio2b  and \verb cLO_FDDQ_BUF_X24 c, frequency pulling drops below 1 MHz. 

So this specification looks fine, testbench shown:


How does portAdapter work?


Frequency pulling mentions coupling (crosstalk) between different blocks on chip and the VCO.
%----------------------------------------------------------------------------------------
\subsection{PTAT and CTAT, the temperature independant Vbias for cascode}

% \verb c c

TODO Simulate and show results of temperature sweep.


%----------------------------------------------------------------------------------------
\subsection{Effect of too high tail capacitance}

\begin{figure}[!ht]
	\includegraphics[width=\linewidth]{Figures/squegging.png}
	\caption{Squegging}
	\label{fig:squegging}
\end{figure}

Instability Low frequency amplitude modulation squegging. This was the issue with the original class C with tail current. Is it important for new topology where current bias is PMOS.

\subsection{PLL theory - PLL Lock Detect and calibration}

About PLL Lock Detect:

\begin{info} % Information block
	The ability for a PLL to reliably indicate when it is in lock is critical for many applications. An ideal lock detect circuit gives a high indication when the PLL is locked and a low indication when the PLL is unlocked. When VCO calibration finishes it can be indicative if the lock is detected. 
\end{info}

PLL VCO calibration usually goes as amplitude frequency and then amplitude calibration, or the other way?

Divide and Conquer algorithm


\begin{figure}[!ht]
	\includegraphics[width=\linewidth]{Figures/PLL_basics_regarding_VCO.png}
	\caption{Gain for VCO and  other blocks inside of PLL}
	\label{fig:PLL_basics_regarding_VCO	}
\end{figure}

\begin{info}
	For the VCO, the noise is suppressed below the loop bandwidth frequency and unshaped above the loop bandwidth.
\end{info}


% 50 or 100 ns for the SAW filter

% differential instead of pseudo differential 

% source degeneration


% \begin{question}[\itshape What kind of output is needed ?]
% 	s
% \end{question}
